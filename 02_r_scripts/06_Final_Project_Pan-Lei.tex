% Options for packages loaded elsewhere
\PassOptionsToPackage{unicode}{hyperref}
\PassOptionsToPackage{hyphens}{url}
%
\documentclass[
]{article}
\usepackage{amsmath,amssymb}
\usepackage{lmodern}
\usepackage{iftex}
\ifPDFTeX
  \usepackage[T1]{fontenc}
  \usepackage[utf8]{inputenc}
  \usepackage{textcomp} % provide euro and other symbols
\else % if luatex or xetex
  \usepackage{unicode-math}
  \defaultfontfeatures{Scale=MatchLowercase}
  \defaultfontfeatures[\rmfamily]{Ligatures=TeX,Scale=1}
\fi
% Use upquote if available, for straight quotes in verbatim environments
\IfFileExists{upquote.sty}{\usepackage{upquote}}{}
\IfFileExists{microtype.sty}{% use microtype if available
  \usepackage[]{microtype}
  \UseMicrotypeSet[protrusion]{basicmath} % disable protrusion for tt fonts
}{}
\makeatletter
\@ifundefined{KOMAClassName}{% if non-KOMA class
  \IfFileExists{parskip.sty}{%
    \usepackage{parskip}
  }{% else
    \setlength{\parindent}{0pt}
    \setlength{\parskip}{6pt plus 2pt minus 1pt}}
}{% if KOMA class
  \KOMAoptions{parskip=half}}
\makeatother
\usepackage{xcolor}
\usepackage[margin=1in]{geometry}
\usepackage{color}
\usepackage{fancyvrb}
\newcommand{\VerbBar}{|}
\newcommand{\VERB}{\Verb[commandchars=\\\{\}]}
\DefineVerbatimEnvironment{Highlighting}{Verbatim}{commandchars=\\\{\}}
% Add ',fontsize=\small' for more characters per line
\usepackage{framed}
\definecolor{shadecolor}{RGB}{248,248,248}
\newenvironment{Shaded}{\begin{snugshade}}{\end{snugshade}}
\newcommand{\AlertTok}[1]{\textcolor[rgb]{0.94,0.16,0.16}{#1}}
\newcommand{\AnnotationTok}[1]{\textcolor[rgb]{0.56,0.35,0.01}{\textbf{\textit{#1}}}}
\newcommand{\AttributeTok}[1]{\textcolor[rgb]{0.77,0.63,0.00}{#1}}
\newcommand{\BaseNTok}[1]{\textcolor[rgb]{0.00,0.00,0.81}{#1}}
\newcommand{\BuiltInTok}[1]{#1}
\newcommand{\CharTok}[1]{\textcolor[rgb]{0.31,0.60,0.02}{#1}}
\newcommand{\CommentTok}[1]{\textcolor[rgb]{0.56,0.35,0.01}{\textit{#1}}}
\newcommand{\CommentVarTok}[1]{\textcolor[rgb]{0.56,0.35,0.01}{\textbf{\textit{#1}}}}
\newcommand{\ConstantTok}[1]{\textcolor[rgb]{0.00,0.00,0.00}{#1}}
\newcommand{\ControlFlowTok}[1]{\textcolor[rgb]{0.13,0.29,0.53}{\textbf{#1}}}
\newcommand{\DataTypeTok}[1]{\textcolor[rgb]{0.13,0.29,0.53}{#1}}
\newcommand{\DecValTok}[1]{\textcolor[rgb]{0.00,0.00,0.81}{#1}}
\newcommand{\DocumentationTok}[1]{\textcolor[rgb]{0.56,0.35,0.01}{\textbf{\textit{#1}}}}
\newcommand{\ErrorTok}[1]{\textcolor[rgb]{0.64,0.00,0.00}{\textbf{#1}}}
\newcommand{\ExtensionTok}[1]{#1}
\newcommand{\FloatTok}[1]{\textcolor[rgb]{0.00,0.00,0.81}{#1}}
\newcommand{\FunctionTok}[1]{\textcolor[rgb]{0.00,0.00,0.00}{#1}}
\newcommand{\ImportTok}[1]{#1}
\newcommand{\InformationTok}[1]{\textcolor[rgb]{0.56,0.35,0.01}{\textbf{\textit{#1}}}}
\newcommand{\KeywordTok}[1]{\textcolor[rgb]{0.13,0.29,0.53}{\textbf{#1}}}
\newcommand{\NormalTok}[1]{#1}
\newcommand{\OperatorTok}[1]{\textcolor[rgb]{0.81,0.36,0.00}{\textbf{#1}}}
\newcommand{\OtherTok}[1]{\textcolor[rgb]{0.56,0.35,0.01}{#1}}
\newcommand{\PreprocessorTok}[1]{\textcolor[rgb]{0.56,0.35,0.01}{\textit{#1}}}
\newcommand{\RegionMarkerTok}[1]{#1}
\newcommand{\SpecialCharTok}[1]{\textcolor[rgb]{0.00,0.00,0.00}{#1}}
\newcommand{\SpecialStringTok}[1]{\textcolor[rgb]{0.31,0.60,0.02}{#1}}
\newcommand{\StringTok}[1]{\textcolor[rgb]{0.31,0.60,0.02}{#1}}
\newcommand{\VariableTok}[1]{\textcolor[rgb]{0.00,0.00,0.00}{#1}}
\newcommand{\VerbatimStringTok}[1]{\textcolor[rgb]{0.31,0.60,0.02}{#1}}
\newcommand{\WarningTok}[1]{\textcolor[rgb]{0.56,0.35,0.01}{\textbf{\textit{#1}}}}
\usepackage{graphicx}
\makeatletter
\def\maxwidth{\ifdim\Gin@nat@width>\linewidth\linewidth\else\Gin@nat@width\fi}
\def\maxheight{\ifdim\Gin@nat@height>\textheight\textheight\else\Gin@nat@height\fi}
\makeatother
% Scale images if necessary, so that they will not overflow the page
% margins by default, and it is still possible to overwrite the defaults
% using explicit options in \includegraphics[width, height, ...]{}
\setkeys{Gin}{width=\maxwidth,height=\maxheight,keepaspectratio}
% Set default figure placement to htbp
\makeatletter
\def\fps@figure{htbp}
\makeatother
\setlength{\emergencystretch}{3em} % prevent overfull lines
\providecommand{\tightlist}{%
  \setlength{\itemsep}{0pt}\setlength{\parskip}{0pt}}
\setcounter{secnumdepth}{-\maxdimen} % remove section numbering
\ifLuaTeX
  \usepackage{selnolig}  % disable illegal ligatures
\fi
\IfFileExists{bookmark.sty}{\usepackage{bookmark}}{\usepackage{hyperref}}
\IfFileExists{xurl.sty}{\usepackage{xurl}}{} % add URL line breaks if available
\urlstyle{same} % disable monospaced font for URLs
\hypersetup{
  pdftitle={HUDM6026 Final Project},
  pdfauthor={Chenguang Pan \& Seng Lei},
  hidelinks,
  pdfcreator={LaTeX via pandoc}}

\title{HUDM6026 Final Project}
\author{Chenguang Pan \& Seng Lei}
\date{April 28, 2023}

\begin{document}
\maketitle

\hypertarget{introduction}{%
\subsection{1.0 Introduction}\label{introduction}}

High School Longitudinal Study of 2009(HSLS:09) is a nationally
representative, longitudinal study of 23,000+ ninth graders from 944
schools in 2009. It provides comprehensive information about student's
background, academic performance in both high school and college,
personal attitudes towards study and school, etc. Therefore, this
current project uses the HSLS:09 open dataset.

This study analyzed a large dataset consisting of 23,503 observations
and 9,614 variables to investigate the potential relationship between
ninth-grade mathematics foundation and future achievement in STEM
fields. To accomplish this objective, a simple linear regression model
was employed, which allowed for the estimation of the effect of math
proficiency on overall GPA in STEM courses throughout high school. The
standardized mathematics assessment of algebraic reasoning, administered
during the first semester of grade 9, was utilized to measure students'
mathematical abilities at the onset of high school. In turn, the overall
GPA in STEM courses was used as a metric of academic performance in STEM
subjects throughout the high school years.

After some data cleaning, we kept 199,948 observations for analysis. The
simple linear model is: \[y_i = \beta_0 + \beta_1\times x_i + e_i ,\]
where \(y_i\) is the estimated individual outcome for overall STEM GPA,
\(x_i\) is the student's mathematics assessment score, and \(e_i\) is
the measurement error.

\hypertarget{population-data-descriptions}{%
\subsection{2.0 Population data
descriptions}\label{population-data-descriptions}}

As a simulated study, we treated cleaned dataset as the population,
\emph{N}=19948. The mean and standard deviation for the dependent
variable are 2.440 and .934. And 51.250 and 10.031 for the predictor.
The correlation coefficient between these two variables is .567.

\begin{Shaded}
\begin{Highlighting}[]
\SpecialCharTok{\textgreater{}}\NormalTok{ model\_lm }\OtherTok{\textless{}{-}} \FunctionTok{lm}\NormalTok{(X3TGPASTEM }\SpecialCharTok{\textasciitilde{}}\NormalTok{ X1TXMTSCOR, }\AttributeTok{data =}\NormalTok{ hsls\_sub)}
\SpecialCharTok{\textgreater{}} \FunctionTok{summary}\NormalTok{(model\_lm)}

\NormalTok{Call}\SpecialCharTok{:}
\FunctionTok{lm}\NormalTok{(}\AttributeTok{formula =}\NormalTok{ X3TGPASTEM }\SpecialCharTok{\textasciitilde{}}\NormalTok{ X1TXMTSCOR, }\AttributeTok{data =}\NormalTok{ hsls\_sub)}

\NormalTok{Residuals}\SpecialCharTok{:}
\NormalTok{     Min       1Q   Median       3Q      Max }
\SpecialCharTok{{-}}\FloatTok{2.82822} \SpecialCharTok{{-}}\FloatTok{0.50345}  \FloatTok{0.05364}  \FloatTok{0.55167}  \FloatTok{2.70153} 

\NormalTok{Coefficients}\SpecialCharTok{:}
\NormalTok{             Estimate Std. Error t value }\FunctionTok{Pr}\NormalTok{(}\SpecialCharTok{\textgreater{}}\ErrorTok{|}\NormalTok{t}\SpecialCharTok{|}\NormalTok{)    }
\NormalTok{(Intercept) }\SpecialCharTok{{-}}\FloatTok{0.264846}   \FloatTok{0.028358}  \SpecialCharTok{{-}}\FloatTok{9.339}   \SpecialCharTok{\textless{}}\FloatTok{2e{-}16} \SpecialCharTok{**}\ErrorTok{*}
\NormalTok{X1TXMTSCOR   }\FloatTok{0.052792}   \FloatTok{0.000543}  \FloatTok{97.220}   \SpecialCharTok{\textless{}}\FloatTok{2e{-}16} \SpecialCharTok{**}\ErrorTok{*}
\SpecialCharTok{{-}{-}{-}}
\NormalTok{Signif. codes}\SpecialCharTok{:}  \DecValTok{0} \StringTok{\textquotesingle{}***\textquotesingle{}} \FloatTok{0.001} \StringTok{\textquotesingle{}**\textquotesingle{}} \FloatTok{0.01} \StringTok{\textquotesingle{}*\textquotesingle{}} \FloatTok{0.05} \StringTok{\textquotesingle{}.\textquotesingle{}} \FloatTok{0.1} \StringTok{\textquotesingle{} \textquotesingle{}} \DecValTok{1}

\NormalTok{Residual standard error}\SpecialCharTok{:} \FloatTok{0.7693}\NormalTok{ on }\DecValTok{19946}\NormalTok{ degrees of freedom}
\NormalTok{Multiple R}\SpecialCharTok{{-}}\NormalTok{squared}\SpecialCharTok{:}  \FloatTok{0.3215}\NormalTok{,    Adjusted R}\SpecialCharTok{{-}}\NormalTok{squared}\SpecialCharTok{:}  \FloatTok{0.3215} 
\NormalTok{F}\SpecialCharTok{{-}}\NormalTok{statistic}\SpecialCharTok{:}  \DecValTok{9452}\NormalTok{ on }\DecValTok{1}\NormalTok{ and }\DecValTok{19946}\NormalTok{ DF,  p}\SpecialCharTok{{-}}\NormalTok{value}\SpecialCharTok{:} \ErrorTok{\textless{}} \FloatTok{2.2e{-}16}
\end{Highlighting}
\end{Shaded}

The simple linear regression model presented that the overall model can
explain the 32.15\% of the variance in outcome, \(F(1,19946)=9452\),
\(p < .001\). One score increase in 9th grader's math assessment will be
associated with .053 increase in overall STEM GPA and this relation is
statistically significant, \(\beta_1=.053\), \(p <.001\). The expected
value of overall STEM GPA (i.e., \(\beta_0\)) when student gets zero in
math assessment is \(-.265\), \(p <.001\). The negative GPA does not
make any sense, but we ignored this issue and move on the study.

\hypertarget{writing-r-functions}{%
\subsection{3.0 Writing R functions}\label{writing-r-functions}}

\begin{Shaded}
\begin{Highlighting}[]
\SpecialCharTok{\textgreater{}}\NormalTok{ dat\_gen }\OtherTok{\textless{}{-}} \ControlFlowTok{function}\NormalTok{(}\AttributeTok{size=} \DecValTok{500}\NormalTok{,  }\CommentTok{\# smaple size}
\SpecialCharTok{+}\NormalTok{                     betas,      }\CommentTok{\# a numeric array of betas}
\SpecialCharTok{+}\NormalTok{                     iv\_mean,    }\CommentTok{\# predictor\textquotesingle{}s mean}
\SpecialCharTok{+}\NormalTok{                     iv\_var,     }\CommentTok{\# predictor\textquotesingle{}s variance}
\SpecialCharTok{+}\NormalTok{                     error\_sd)\{  }\CommentTok{\# residuals\textquotesingle{} sd}
\SpecialCharTok{+}   \CommentTok{\# data mainly are generated from a normal distribution \textasciitilde{} N(iv\_mean, iv\_sd)}
\SpecialCharTok{+}\NormalTok{   X }\OtherTok{\textless{}{-}} \FunctionTok{rnorm}\NormalTok{(size, }\AttributeTok{mean =}\NormalTok{ iv\_mean, }\AttributeTok{sd=} \FunctionTok{sqrt}\NormalTok{(iv\_var))}
\SpecialCharTok{+}\NormalTok{   X\_aug }\OtherTok{\textless{}{-}} \FunctionTok{cbind}\NormalTok{(}\DecValTok{1}\NormalTok{, X)}
\SpecialCharTok{+}   \CommentTok{\# residuals are generated from \textasciitilde{}N(0, sd)}
\SpecialCharTok{+}\NormalTok{   Error }\OtherTok{\textless{}{-}} \FunctionTok{rnorm}\NormalTok{(size, }\AttributeTok{mean=}\DecValTok{0}\NormalTok{, }\AttributeTok{sd=}\NormalTok{error\_sd)}
\SpecialCharTok{+}   \CommentTok{\# based on the parameters to generate the outcomes}
\SpecialCharTok{+}\NormalTok{   Y }\OtherTok{\textless{}{-}}\NormalTok{ X\_aug }\SpecialCharTok{\%*\%} \FunctionTok{as.matrix}\NormalTok{(betas) }\SpecialCharTok{+}\NormalTok{ Error}
\SpecialCharTok{+}\NormalTok{   out }\OtherTok{\textless{}{-}} \FunctionTok{cbind}\NormalTok{(Y, X)}
\SpecialCharTok{+}   \FunctionTok{colnames}\NormalTok{(out) }\OtherTok{\textless{}{-}} \FunctionTok{c}\NormalTok{(}\StringTok{"Y"}\NormalTok{, }\StringTok{"X1"}\NormalTok{)}
\SpecialCharTok{+}   \FunctionTok{return}\NormalTok{(}\FunctionTok{as.data.frame}\NormalTok{(out))}
\SpecialCharTok{+}\NormalTok{ \}}
\end{Highlighting}
\end{Shaded}

The data generation function takes the sample size, regression
coefficients, predictors' mean and variance, and the standard deviation
of residual as input. It returns a simulated dataset in
\texttt{dataframe} format with the outcome in the first column and
predictor in the second.

\begin{Shaded}
\begin{Highlighting}[]
\SpecialCharTok{\textgreater{}}\NormalTok{ reg }\OtherTok{\textless{}{-}} \ControlFlowTok{function}\NormalTok{(ds) \{}
\SpecialCharTok{+}\NormalTok{   x }\OtherTok{\textless{}{-}} \FunctionTok{as.matrix}\NormalTok{(ds[,}\DecValTok{2}\NormalTok{])}
\SpecialCharTok{+}\NormalTok{   y }\OtherTok{\textless{}{-}} \FunctionTok{as.matrix}\NormalTok{(ds[,}\DecValTok{1}\NormalTok{])}
\SpecialCharTok{+}\NormalTok{   y\_cen }\OtherTok{\textless{}{-}} \FunctionTok{apply}\NormalTok{(y, }\DecValTok{2}\NormalTok{, }\ControlFlowTok{function}\NormalTok{(x) x}\SpecialCharTok{{-}}\FunctionTok{mean}\NormalTok{(x))}
\SpecialCharTok{+}\NormalTok{   x\_cen }\OtherTok{\textless{}{-}} \FunctionTok{apply}\NormalTok{(x, }\DecValTok{2}\NormalTok{, }\ControlFlowTok{function}\NormalTok{(x) x}\SpecialCharTok{{-}}\FunctionTok{mean}\NormalTok{(x))}
\SpecialCharTok{+}   \CommentTok{\# the OLS method}
\SpecialCharTok{+}\NormalTok{   b1 }\OtherTok{\textless{}{-}} \FunctionTok{sum}\NormalTok{(x\_cen}\SpecialCharTok{*}\NormalTok{y\_cen)}\SpecialCharTok{/}\FunctionTok{sum}\NormalTok{(x\_cen}\SpecialCharTok{\^{}}\DecValTok{2}\NormalTok{)}
\SpecialCharTok{+}\NormalTok{   b0 }\OtherTok{\textless{}{-}} \FunctionTok{mean}\NormalTok{(y }\SpecialCharTok{{-}}\NormalTok{ x}\SpecialCharTok{*}\NormalTok{b1)}
\SpecialCharTok{+}\NormalTok{   y\_hat }\OtherTok{\textless{}{-}}\NormalTok{ b0 }\SpecialCharTok{+}\NormalTok{ x}\SpecialCharTok{*}\NormalTok{b1}
\SpecialCharTok{+}\NormalTok{   sse }\OtherTok{\textless{}{-}} \FunctionTok{sum}\NormalTok{((y}\SpecialCharTok{{-}}\NormalTok{y\_hat)}\SpecialCharTok{\^{}}\DecValTok{2}\NormalTok{)}
\SpecialCharTok{+}\NormalTok{   sig\_sq }\OtherTok{\textless{}{-}}\NormalTok{ sse}\SpecialCharTok{/}\NormalTok{(}\FunctionTok{nrow}\NormalTok{(x)}\SpecialCharTok{{-}}\DecValTok{2}\NormalTok{)}
\SpecialCharTok{+}   \CommentTok{\# the alternative method}
\SpecialCharTok{+}\NormalTok{   b1\_a }\OtherTok{\textless{}{-}} \FunctionTok{sum}\NormalTok{(y\_cen}\SpecialCharTok{/}\NormalTok{x\_cen)}\SpecialCharTok{/}\FunctionTok{nrow}\NormalTok{(x)}
\SpecialCharTok{+}\NormalTok{   b0\_a }\OtherTok{\textless{}{-}} \FunctionTok{mean}\NormalTok{(y }\SpecialCharTok{{-}}\NormalTok{ x}\SpecialCharTok{*}\NormalTok{b1\_a)}
\SpecialCharTok{+}\NormalTok{   y\_hat\_a }\OtherTok{\textless{}{-}}\NormalTok{ b0\_a }\SpecialCharTok{+}\NormalTok{ x}\SpecialCharTok{*}\NormalTok{b1\_a}
\SpecialCharTok{+}\NormalTok{   sse\_a }\OtherTok{\textless{}{-}} \FunctionTok{sum}\NormalTok{((y}\SpecialCharTok{{-}}\NormalTok{y\_hat\_a)}\SpecialCharTok{\^{}}\DecValTok{2}\NormalTok{)}
\SpecialCharTok{+}\NormalTok{   sig\_sq\_a }\OtherTok{\textless{}{-}}\NormalTok{ sse\_a}\SpecialCharTok{/}\FunctionTok{nrow}\NormalTok{(x)}
\SpecialCharTok{+}\NormalTok{   out\_ }\OtherTok{\textless{}{-}} \FunctionTok{cbind}\NormalTok{(b0, b1, sig\_sq, b0\_a, b1\_a,sig\_sq\_a)}
\SpecialCharTok{+}   \FunctionTok{return}\NormalTok{(out\_)}
\SpecialCharTok{+}\NormalTok{ \}}
\end{Highlighting}
\end{Shaded}

The estimation function takes the simulated data frame as input, and
returns the estimated \(\beta_0\), \(\beta_1\), residual's variance
\(\sigma^2\) from both least square and alternative methods.

\hypertarget{monte-carlo-simulation}{%
\subsection{4.0 Monte Carlo Simulation}\label{monte-carlo-simulation}}

The basic idea behind the Monte Carlo simulation is that one can draw a
large number of random samples from a probability distribution
representing the population being studied, and then these samples are
used to estimate its statistical properties. This project used Monte
Carlo method to draw 1000 random samples with the size of 40 by using
the \texttt{dat\_gen()} function above.

\begin{Shaded}
\begin{Highlighting}[]
\SpecialCharTok{\textgreater{}}\NormalTok{ R }\OtherTok{\textless{}{-}} \DecValTok{1000}
\SpecialCharTok{\textgreater{}} \FunctionTok{set.seed}\NormalTok{(}\DecValTok{666}\NormalTok{)}
\SpecialCharTok{\textgreater{}} \CommentTok{\# randomly generate 1000 samples}
\ErrorTok{\textgreater{}}\NormalTok{ dat\_list }\OtherTok{\textless{}{-}} \FunctionTok{replicate}\NormalTok{(}\AttributeTok{n =}\NormalTok{ R,}
\SpecialCharTok{+}                       \AttributeTok{expr =} \FunctionTok{dat\_gen}\NormalTok{(}\AttributeTok{size =} \DecValTok{40}\NormalTok{,}
\SpecialCharTok{+}                                      \AttributeTok{betas =} \FunctionTok{c}\NormalTok{(}\SpecialCharTok{{-}}\FloatTok{0.265}\NormalTok{,}\FloatTok{0.053}\NormalTok{),}
\SpecialCharTok{+}                                      \AttributeTok{iv\_mean =} \FloatTok{51.24985}\NormalTok{, }\AttributeTok{iv\_var =} \FloatTok{100.6209}\NormalTok{,}
\SpecialCharTok{+}                                      \AttributeTok{error\_sd =} \FloatTok{0.7693}\NormalTok{),}
\SpecialCharTok{+}                       \AttributeTok{simplify =} \ConstantTok{FALSE}\NormalTok{)}
\SpecialCharTok{\textgreater{}} \CommentTok{\# estimated the simple regression model on each sample}
\ErrorTok{\textgreater{}}\NormalTok{ estimates }\OtherTok{\textless{}{-}} \FunctionTok{sapply}\NormalTok{(}\AttributeTok{X =}\NormalTok{ dat\_list,}
\SpecialCharTok{+}                     \AttributeTok{FUN =}\NormalTok{ reg,}
\SpecialCharTok{+}                     \AttributeTok{simplify =} \ConstantTok{TRUE}\NormalTok{)}
\SpecialCharTok{\textgreater{}}\NormalTok{ estimates }\OtherTok{\textless{}{-}} \FunctionTok{t}\NormalTok{(estimates)}
\SpecialCharTok{\textgreater{}} \FunctionTok{colnames}\NormalTok{(estimates) }\OtherTok{\textless{}{-}} \FunctionTok{c}\NormalTok{(}\StringTok{"b0"}\NormalTok{, }\StringTok{"b1"}\NormalTok{, }\StringTok{"sig\_sq"}\NormalTok{, }\StringTok{"b0\_a"}\NormalTok{, }\StringTok{"b1\_a"}\NormalTok{, }\StringTok{"sig\_sq\_a"}\NormalTok{)}
\SpecialCharTok{\textgreater{}}\NormalTok{ (estimates\_hat\_mean }\OtherTok{\textless{}{-}} \FunctionTok{round}\NormalTok{(}\FunctionTok{apply}\NormalTok{(estimates,}\DecValTok{2}\NormalTok{,mean),}\DecValTok{3}\NormalTok{))}
\NormalTok{      b0       b1   sig\_sq     b0\_a     b1\_a sig\_sq\_a }
  \SpecialCharTok{{-}}\FloatTok{0.265}    \FloatTok{0.053}    \FloatTok{0.585}   \SpecialCharTok{{-}}\FloatTok{5.143}    \FloatTok{0.146}  \FloatTok{121.727} 
\SpecialCharTok{\textgreater{}}\NormalTok{ (estimates\_hat\_sd }\OtherTok{\textless{}{-}} \FunctionTok{round}\NormalTok{(}\FunctionTok{apply}\NormalTok{(estimates,}\DecValTok{2}\NormalTok{,sd),}\DecValTok{3}\NormalTok{))}
\NormalTok{      b0       b1   sig\_sq     b0\_a     b1\_a sig\_sq\_a }
   \FloatTok{0.645}    \FloatTok{0.012}    \FloatTok{0.131}   \FloatTok{59.657}    \FloatTok{1.148}  \FloatTok{935.836} 
\end{Highlighting}
\end{Shaded}


\end{document}
